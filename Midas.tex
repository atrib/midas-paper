\documentclass[letterpaper,twocolumn,10pt]{article}
\usepackage{usenix-2020-09}

\usepackage{adjustbox}
\usepackage{amsmath}
\usepackage{filecontents}
\usepackage{tikz}
\usepackage{xspace}
\usepackage{listings}
\usepackage{xcolor}
\usepackage{textcomp}
\usepackage{graphicx}
\usepackage{caption}
\usepackage{subcaption}
\usepackage{todonotes}
\usepackage{multirow}
\usepackage{booktabs}
\usepackage{paralist}
\usepackage{enumitem}
\usepackage{tabulary}
\usepackage{prettyref}
\usepackage{verbatim}
\usepackage{balance}
\usepackage{tabularx}
\usepackage{algpseudocode}
\usepackage[available,functional,reproduced]{usenixbadges}

%-------------------------------------------------------------------------------
\begin{document}
%-------------------------------------------------------------------------------

\newenvironment{minilisting}
  {\minipage[b]{\linewidth}\verbatim}
  {\endverbatim\endminipage}

\lstdefinestyle{cstyle}{
  basicstyle=\footnotesize\ttfamily,
  keywordstyle=\color{black!85}\bfseries,
  keywordstyle=[2]\color{black!85}\bfseries\emph,
  showstringspaces=false,
  language={C},
  breaklines=false,
  mathescape=true,
  escapechar={@}
}
\lstdefinestyle{inline}{
  style=cstyle,
  mathescape=false,
  breaklines=true,
  keywordstyle=,
  keywordstyle=[2],
  extendedchars=true,
  basicstyle=\ttfamily\small
}
\newcommand{\Code}[1]{\lstinline[style=inline,breaklines=false]@#1@}
\let\realparagraph\paragraph
\let\paragraph\relax
\newcommand{\paragraph}[1]{\textbf{#1.}}
\newcolumntype{Q}{>{\centering\arraybackslash}X}
\newcolumntype{M}[1]{>{\centering\arraybackslash}m{#1}}

\newcommand\tocttou[0]{TOCTTOU\xspace}
\newcommand\midas[0]{Midas\xspace}

\newrefformat{cha}{\hyperref[#1]{Chapter~\ref*{#1}}}
\newrefformat{sec}{\hyperref[#1]{Section~\ref*{#1}}}
\newrefformat{sub}{\hyperref[#1]{Section~\ref*{#1}}}
\newrefformat{tab}{\hyperref[#1]{Table~\ref*{#1}}}
\newrefformat{fig}{\hyperref[#1]{Figure~\ref*{#1}}}
\newrefformat{line}{\hyperref[#1]{line~\ref*{#1}}}
\newrefformat{lst}{\hyperref[#1]{Listing~\ref*{#1}}}
\newrefformat{pat}{\hyperref[#1]{Patch~\ref*{#1}}}
\newrefformat{alg}{\hyperref[#1]{Algorithm~\ref*{#1}}}
\renewcommand{\sectionautorefname}{Section}
\let\subsectionautorefname\sectionautorefname
\let\subsubsectionautorefname\sectionautorefname

\renewcommand\itemautorefname{Attack}

\newcommand\mat[1]{\noindent{\color{blue} {\bf \fbox{Mat}} {\it#1}}}
\newcommand\atri[1]{\noindent{\color{red} {\bf \fbox{AB}} {\it#1}}}
\newcommand\UT[1]{\noindent{\color{violet} {\bf \fbox{UT}} {\it#1}}}


\newcommand{\doinumber}{5753026}
\newcommand{\zenodorecord}[0]{\url{https://zenodo.org/record/\doinumber}}
  \newcommand{\zenododoi}[0]{10.5281/zenodo.\doinumber\xspace}
  
% \newcommand\mat[1]{}
% \newcommand\atri[1]{}
% \newcommand\UT[1]{}

\newcommand{\footremember}[2]{%
   \footnote{#2}
    \newcounter{#1}
    \setcounter{#1}{\value{footnote}}%
}
\newcommand{\footrecall}[1]{%
    \footnotemark[\value{#1}]%
}

%don't want date printed
\date{}

% make title bold and 14 pt font (Latex default is non-bold, 16 pt)
\title{\Large \bf \midas: Systematic Kernel TOCTTOU Protection}
% \title{TocSens: TOCTTOU-sensitive Sunglasses for user access}

%for single author (just remove % characters)
\author{
{\rm Atri Bhattacharyya} \\
EPFL
% copy the following lines to add more authors
\and
{\rm Uros Tesic} \thanks{This work was done during the author's time at EPFL.}\\
Nvidia
\and
{\rm Mathias Payer}\\
EPFL
} % end author
% \thanks{asdf}
\maketitle

%-------------------------------------------------------------------------------
\begin{abstract}
%-------------------------------------------------------------------------------
Double-fetch bugs are a plague across all major operating system kernels. They
occur when data is fetched twice across the user/kernel trust boundary while
allowing concurrent modification. Such bugs enable an attacker to illegally
access memory, cause denial of service, or to escalate privileges.
%
So far, the only protection against double-fetch bugs is to detect and fix them.
However, they remain incredibly hard to find.
%
Similarly,
they fundamentally prohibit efficient, kernel-based stateful system call filtering.
Thus, we propose \emph{\midas} to mitigate double-fetch bugs. \midas creates on-demand
snapshots and copies of accessed data, enforcing our key invariant
that throughout a system call's lifetime, every read to a userspace object
will return the same value.

\midas shows no noticeable drop in performance when evaluated on compute-bound
workloads. On system call heavy workloads, \midas incurs~0.2--14\%
performance overhead, while protecting the kernel
against any \tocttou attacks. On average, \midas shows a~$3.4\%$ overhead on
diverse workloads across two benchmark suites.

\end{abstract}

\input{Midas-core}

%%%%%%%%%%%%%%%%%%%%
\section*{Acknowledgements}
%%%%%%%%%%%%%%%%%%%%
The authors thank Marcel Busch and the anonymous reviewers for their careful
feedback and support during the writing of this
paper.
%
This project was partially supported by European Research Council (ERC)
grant No. 850868, DARPA HR001119S0089-AMP-FP-034, and ONR award
N00014-18-1-2674. Any findings are those of the authors and do not necessarily
reflect the views of our sponsors.

%%%%%%%%%%%%%%%%%%%%
% Bibliography
%%%%%%%%%%%%%%%%%%%%
\balance
\bibliographystyle{plain}
\bibliography{Midas}


\appendix
\input{Midas-appendix}

%%%%%%%%%%%%%%%%%%%%%%%%%%%%%%%%%%%%%%%%%%%%%%%%%%%%%%%%%%%%%%%%%%%%%%%%%%%%%%%%
\end{document}
%%%%%%%%%%%%%%%%%%%%%%%%%%%%%%%%%%%%%%%%%%%%%%%%%%%%%%%%%%%%%%%%%%%%%%%%%%%%%%%%

%%  LocalWords:  endnotes includegraphics fread ptr nobj noindent
%%  LocalWords:  pdflatex acks
